\section{Configuration options}
\label{confg_chap}
Table \ref{generics} shows the configuration parameters (VDHL generics) exposed by \textit{apb\_wrapper.vhd}. 
\\
\begin{table}[H]
	\caption{Configuration options (VHDL ports)}
	\label{generics}
	\centering
	\begin{small}
		\begin{tabular}{|l|p{6cm}|l|l|}
			\hline
			\textbf{Generic} & \textbf{Function}  & \textbf{Allowed range}  & \textbf{Default}\\
			\hline
			PADDR  & APB base address &0 to 16\#fff\# & 0\\
			\hline
			PMASK  & APB address mask &0 to 16\#fff\# & 16\#fff\#\\
			\hline
			PINDEX & APB slave index &0 to 16\#fff\# & 16\#fff\#\\
			\hline
			PIRQ   & APB interrupt index &0 to 16\#fff\# & 16\#fff\#\\
			\hline
			LANES\_NUMBER & Lanes of the processor & 1 to 5 & 2\\
			\hline
			REGISTER\_OUTPUT & If set to 1, the STALL outpus are registered to improve timing & 0 or 1 & 0 \\
			\hline
			MIN\_STAGGERING\_INIT & If the minimum staggering threshold is not changed through the software, its value will be this parameter& 5 to 32740 & 20 \\
			\hline
                        EN\_CYCLES\_LIMIT & Maximum number of cycles allowed between setting to 1 one of the critical section registers \ref{critical1} \ref{critical2} and the other before raising an interrupt & ALL & 500 \\
			\hline
		\end{tabular}
	\end{small}
\end{table}

\hspace{2cm}
